\documentclass{article}
\usepackage{listings}
\usepackage[cm]{fullpage}
\begin{document}
\title{Initial Documentation for the Out and About Project}
\author{Felix Farquharson - fef}
\maketitle{}

\tableofcontents{}

\section{Event Creation program}
I decided to do this part of the project in the C programming 
language because it seemed the most linear part. There is a seperate
file for the menu. The header files are stored centrally. It's built 
by a general make file that builds the whole program.

\subsection{The main.c file}
The main.c file contains in infinate loop that will only ever end
when the menu returns 1, which means it's exited.
\lstinputlisting{../src/event_creation/main.c}

\subsection{The menu.c file}
The menu.c file contains all of the logic to display the menu and
to take input from the user. It presents a formatted output and
waits for the user to enter a letter before evaluating that letter
and running the relevant code block.
\lstinputlisting{../src/event_creation/menu.c}

\subsection{The menu.h file}
The menu.h file is in seperate directory to keep the src folder tidy.
It's used from the main.c file to include the menu.
\lstinputlisting{../src/event_creation/h/menu.h}

\section{The Checkpoint Manager program}
The checkpoint manager program is written in java. Java seemed the
easiest of the three languages to write a UI in hence the choice.
There are seperate directories to hold the ui and the data code, 
this helps with organisation. The UI is written in Java Swing.

\subsection{The Main.java file}
The Main.java file checks that the correct arguments have been
given to run the application, it will then store the locations
because they are parsed later. The last thing it does is to 
start an instance of the CheckpointFrame class to display the UI.
\lstinputlisting{../src/checkpoint_manager/gui/Main.java}

\subsection{The CheckpointFrame.java file}
The CheckpointFrame.java file holds the base frame for the UI
it sets the size, title, adds the menubar and action on close.
it passes instances of the medical panel and time panel so they
can be swapped later when the user needs a different screen.
\lstinputlisting{../src/checkpoint_manager/gui/CheckpointFrame.java}

\subsection{The CheckpointMenu.java file}
The CheckpointManu.java file holds all of the buttons for the menu
bar and it assigns listeners defined in another class to the menu.
\lstinputlisting{../src/checkpoint_manager/gui/CheckpointMenu.java}

\subsection{The CheckpointMenuListener.java file}
The CheckpointMenuListener.java contains all the actions to be 
performed when the buttons on the menu are clicked. Most notably
the actions to be performed to change the view from medical view to
time view.
\lstinputlisting{../src/checkpoint_manager/gui/CheckpointMenuListener.java}


\end{document}